% !TEX root = optics_term_paper.tex

% ==================== 摘要 ====================
\begin{abstract}
本文面向刚体高速旋转实验(如贾尼别科夫效应观测)的视觉测量需求,设计并实现了一套基于彩色圆点标记的目标检测与轨迹重建系统。系统在成像端通过分析针孔模型、曝光控制与光照光谱等光学因素,建立了颜色测量的理论框架;在算法端采用 HSV 颜色分割与形态学去噪,结合椭圆拟合实现亚像素级中心定位,并通过时空连续性与运动幅度阈值抑制背景误检。

针对实验中常见的过曝、运动模糊及背景干扰问题,本文提出了\textbf{多段 HSV 阈值并集策略}以适应非均匀光照,并引入\textbf{圆度与面积匹配度双重约束}提升椭圆拟合的可靠性。以三机位同步拍摄的 12.16 实验数据为例(485 帧,约 0.87 s),本文在完成 2D 检测后进一步结合多视图三角测量与刚体配准/优化,重建姿态序列(旋转矩阵/四元数)与 Body 系角速度,并给出相空间轨迹、能量/角动量一致性与理论对比等结果;在拟合阻尼模型参数后,理论与实验的角速度平均绝对误差(MAE)约为 2.28 rad/s,能量耗散趋势与预期一致。

\noindent\textbf{关键词}:彩色标记;HSV 颜色空间;成像光学;曝光与运动模糊;椭圆拟合;视觉测量
\end{abstract}

\vspace{1em}
\noindent\textbf{视频演示}:本项目的实验演示视频可通过以下链接访问:\\
\url{https://yunpan.tongji.edu.cn/link/AA4A5D5E90975E456697869F4800C05170}\\
(文件名:光学课题.mp4,有效期至 2026-03-31)

\tableofcontents
\newpage

% ==================== 1 引言 ====================
\section{引言}

\subsection{研究背景与项目起源}

本项目的核心目标是\textbf{测量三维空间中刚体的姿态角}(即旋转矩阵 $\mathcal{R}(t)$ 或等价的欧拉角/四元数表示),进而提取角速度 $\vec\omega(t)$ 用于动力学分析。在经典力学实验中,对刚体三维姿态与运动轨迹的精确测量是验证物理定律(如角动量守恒、中间轴不稳定定理——即"贾尼别科夫效应")的关键。

传统测量方法存在诸多局限:
\begin{itemize}
    \item \textbf{惯性测量单元(IMU)}:需要固定在刚体上,改变刚体质量分布,影响实验的纯粹性;
    \item \textbf{商业动作捕捉系统}:设备昂贵(数十万元),且多为红外主动标记,不适合教学与普及型实验;
    \item \textbf{机械传感器}:接触式测量,难以用于高速旋转场景。
\end{itemize}

基于计算机视觉的非接触式测量具有\textbf{低成本、高灵活性}的优势。通过在刚体表面粘贴彩色标记点,利用相机采集视频并进行图像处理,可以实现对标记点位置的精确跟踪,进而重建刚体的运动轨迹与姿态。

\subsection{圆点标记的几何与光学优势}

本系统选择\textbf{彩色圆点}作为标记,兼具几何与光学双重优势:

\subsubsection{几何优势:圆$\to$椭圆的透视特性}

圆形标记在透视投影下呈现为\textbf{椭圆},这一几何特性带来以下检测便利:
\begin{enumerate}
    \item \textbf{形状可预测}:无论刚体姿态如何变化,圆形标记的投影始终为椭圆(包括特殊情况下的圆),其数学模型简洁、拟合算法成熟(如 \texttt{cv2.fitEllipse});
    \item \textbf{中心定位稳定}:椭圆中心在透视变换下近似对应圆心的投影位置,便于实现亚像素级定位;
    \item \textbf{运动模糊兼容}:高速旋转引起的运动模糊会使圆点在图像中拉伸为椭圆形拖影,而椭圆拟合仍可提取其几何中心,无需额外的去模糊处理;
    \item \textbf{长宽比编码姿态}:椭圆的长短轴比例隐含了标记平面相对于成像平面的倾斜角度信息,可辅助姿态估计。
\end{enumerate}

\subsubsection{光学优势:颜色区分与鲁棒性}

选择彩色(而非灰度)标记,有如下光学考量:
\begin{enumerate}
    \item \textbf{颜色对比度}:不同颜色在 HSV 空间中占据不同的色调(Hue)区域,即使在光照变化下,色调信息相对稳定,便于区分多个标记点;
    \item \textbf{降低匹配歧义}:多点跟踪时,颜色可作为额外的身份标识,减少帧间匹配的复杂度;
    \item \textbf{反射特性可控}:通过选择哑光材质,可减少镜面反射高光对颜色测量的干扰。
\end{enumerate}

\subsection{工程约束与困难}

本项目的实际工程场景面临以下挑战:
\begin{itemize}
    \item \textbf{高速运动}:刚体旋转角速度可达数 rad/s,导致显著的运动模糊;
    \item \textbf{光照不均匀}:标记点在旋转过程中迎光面/背光面交替,亮度变化剧烈;
    \item \textbf{透视畸变}:圆形标记在成像平面上投影为椭圆,且长宽比随姿态变化;
    \item \textbf{遮挡}:部分标记点可能被刚体自身遮挡,需要算法具备容错能力。
\end{itemize}

\subsection{本文贡献}

本文针对上述挑战,做出以下主要贡献:
\begin{enumerate}
    \item \textbf{光学-算法链路分析}:将颜色检测问题分解为光学成像链路(光源$\to$反射$\to$成像$\to$采样)与算法处理链路(颜色空间$\to$阈值$\to$形态学$\to$拟合),建立清晰的误差传递模型;
    \item \textbf{多段 HSV 阈值机制}:设计支持"追加模式"的 HSV 标定工具,允许同一颜色 ID 对应多个 HSV 范围的并集,有效应对光照变化;
    \item \textbf{增强型椭圆拟合}:引入圆度(Circularity)与面积匹配度(Area Ratio)双重约束,过滤噪声产生的伪椭圆;
    \item \textbf{排除区域机制}:设计"排除模式",允许用户标记背景干扰区域的 HSV 范围,在检测掩码中予以剔除;
    \item \textbf{ROI 选择功能}:支持交互式框选感兴趣区域,减少背景干扰并提升处理效率。
\end{enumerate}
