% !TEX root = optics_term_paper.tex

% ==================== 2 光学基础 ====================
\section{光学基础与成像误差分析}

\begin{quote}
本章目标:把系统性能"往回追"到光学与成像链路,解释\textbf{为什么同一套算法在不同光照/曝光/镜头条件下表现差异巨大}。
\end{quote}

\subsection{成像几何:针孔模型、像素坐标与尺度}

\subsubsection{针孔相机模型}

相机的成像过程可由\textbf{针孔相机模型(Pinhole Camera Model)}近似描述。设空间中一点 $P$ 的三维坐标为 $(X, Y, Z)$(相机坐标系),其在图像平面上的投影点 $(u, v)$ 满足透视投影关系:
\begin{equation}
u = f_x \frac{X}{Z} + c_x, \quad v = f_y \frac{Y}{Z} + c_y
\end{equation}
其中:
\begin{itemize}
    \item $f_x, f_y$ 为以像素为单位的焦距;
    \item $(c_x, c_y)$ 为图像主点坐标(通常接近图像中心)。
\end{itemize}

用齐次坐标表示为:
\begin{equation}
s \begin{pmatrix} u \\ v \\ 1 \end{pmatrix} = \mathbf{K} \begin{pmatrix} X \\ Y \\ Z \end{pmatrix}, \quad \mathbf{K} = \begin{pmatrix} f_x & 0 & c_x \\ 0 & f_y & c_y \\ 0 & 0 & 1 \end{pmatrix}
\end{equation}
矩阵 $\mathbf{K}$ 称为\textbf{相机内参矩阵}。

\subsubsection{透视投影的几何效应}

由透视投影的非线性特性,空间中的\textbf{圆形标记}在图像平面上通常呈现为\textbf{椭圆}。只有当标记平面与成像平面完全平行时,投影才保持为圆。这一几何事实决定了本系统必须采用椭圆拟合而非圆检测。

设标记点直径为 $D$,距相机距离为 $Z$,则其在图像上的直径约为:
\begin{equation}
d_{\text{px}} \approx \frac{f \cdot D}{Z}
\end{equation}
对于本实验典型参数($f \approx 800$ 像素,$D = 2$ cm,$Z = 50$ cm),标记点在图像上的直径约为 $32$ 像素,满足后续轮廓分析与椭圆拟合的精度要求。

\begin{figure}[htbp]
\centering
\includegraphics[width=1.0\textwidth]{pinhole_model.png}
\caption{针孔相机成像模型示意图}
\end{figure}

\subsection{曝光、噪声与运动模糊}

\subsubsection{曝光时间的影响}

曝光时间 $t_{\text{exp}}$ 是影响图像质量的关键参数:
\begin{itemize}
    \item \textbf{过曝(Overexposure)}:当光照强度过高或曝光时间过长时,传感器像素饱和,R/G/B 通道值趋近最大值 255。此时颜色信息丢失,标记区域呈现为近白色,HSV 空间中\textbf{饱和度 S 急剧下降},传统的单一阈值检测失效;
    \item \textbf{欠曝(Underexposure)}:光照不足时,信号微弱而噪声相对增大,导致信噪比(SNR)降低,阈值分割结果不稳定,易产生散点噪声。
\end{itemize}

\subsubsection{运动模糊}

当刚体高速旋转时,标记点在曝光时间内扫过一段弧线。设像素域速度为 $v_{\text{px}}$(像素/秒),则运动模糊长度为:
\begin{equation}
L_{\text{blur}} \approx v_{\text{px}} \cdot t_{\text{exp}}
\end{equation}
运动模糊导致圆形标记的边缘拖影、轮廓拉长,使其在图像中呈现为\textbf{椭圆形甚至条状}形态。这一效应既是挑战(影响拟合精度),也是提示(说明需要放宽长宽比约束以检测侧视/模糊状态的标记)。

本项目将最大长宽比阈值设为 $15.0$,以允许检测严重模糊或侧视角度很大的标记点。

\subsubsection{噪声模型}

相机传感器的噪声主要包括:
\begin{itemize}
    \item \textbf{光子噪声(Photon Noise)}:服从泊松分布,与信号强度平方根成正比;
    \item \textbf{读出噪声(Readout Noise)}:电子电路的固有噪声,与信号无关。
\end{itemize}

信噪比可近似表示为:
\begin{equation}
\text{SNR} \approx \frac{N_{\text{photon}}}{\sqrt{N_{\text{photon}} + N_{\text{readout}}^2}}
\end{equation}
在低光照条件下,读出噪声主导,SNR 降低,导致 HSV 颜色值波动加剧。

\subsection{镜头畸变、色差与白平衡}

\subsubsection{径向畸变}

实际镜头存在\textbf{桶形/枕形畸变},使图像边缘区域的几何位置产生偏移。畸变模型通常写为:
\begin{equation}
\begin{aligned}
x_{\text{distorted}} &= x (1 + k_1 r^2 + k_2 r^4 + \cdots) \\
y_{\text{distorted}} &= y (1 + k_1 r^2 + k_2 r^4 + \cdots)
\end{aligned}
\end{equation}
其中 $r^2 = x^2 + y^2$,$k_1, k_2$ 为径向畸变系数。

对于高精度轨迹重建,需要通过相机标定获取畸变参数并进行\textbf{去畸变处理}。本项目当前未做相机标定,这可能导致图像边缘区域的标记定位存在 1-3 像素的系统误差。

\subsubsection{色差(Chromatic Aberration)}

由于玻璃对不同波长光的折射率不同,红、绿、蓝三种颜色的成像位置略有偏移。这种\textbf{色差}效应在图像边缘更为显著,可能导致彩色标记边缘出现彩色条纹,影响轮廓提取的精度。

\subsubsection{自动白平衡与曝光}

相机的自动白平衡(AWB)与自动曝光(AE)功能会在视频拍摄过程中动态调整,导致:
\begin{itemize}
    \item 同一颜色在不同帧中的 RGB/HSV 值发生漂移;
    \item 标定好的 HSV 阈值失效,检测率下降。
\end{itemize}

\textbf{建议}:在正式实验中\textbf{锁定白平衡与曝光参数},或采用本项目的\textbf{多段 HSV 范围机制}以适应一定范围内的漂移。

\subsection{光照光谱、材料反射与颜色测量}

\subsubsection{颜色形成的物理机制}

相机记录的颜色并非物体的固有属性,而是\textbf{光源光谱、物体反射率与传感器响应}三者共同作用的结果。设传感器某通道(如红色通道)的响应为 $I_R$,则:
\begin{equation}
I_R \propto \int_{\lambda} E(\lambda) \cdot R(\lambda) \cdot S_R(\lambda) \, d\lambda
\end{equation}
其中:
\begin{itemize}
    \item $E(\lambda)$:光源的光谱功率分布(SPD);
    \item $R(\lambda)$:物体表面在波长 $\lambda$ 处的反射率;
    \item $S_R(\lambda)$:传感器红色通道的光谱响应函数。
\end{itemize}

这解释了\textbf{为什么同一颜色在不同灯光下会呈现不同的 RGB 值}:日光(色温约 5500K)与白炽灯(色温约 2700K)的光谱分布差异显著,即使物体反射率不变,传感器的响应也会不同。

\subsubsection{材料反射特性}

标记点的材质对颜色测量有重要影响:
\begin{itemize}
    \item \textbf{光面材料(Glossy)}:具有镜面反射成分,在特定角度产生高光(Specular Highlight),导致局部区域过曝;
    \item \textbf{哑光材料(Matte)}:漫反射为主,颜色更均匀,但亮度较低。
\end{itemize}

本项目推荐使用\textbf{哑光贴纸或喷砂膜}作为标记材料,以减少高光干扰。

\subsection{为什么使用 HSV:从光学变化到算法鲁棒性}

\subsubsection{RGB 与 HSV 颜色空间对比}

\textbf{RGB 颜色空间}的三个通道(红、绿、蓝)高度相关且均受亮度影响。当光照增强时,R、G、B 值同时增加,导致难以用固定阈值区分不同颜色。

\textbf{HSV 颜色空间}将颜色分解为:
\begin{itemize}
    \item \textbf{H(Hue,色调)}:对应光谱波长,取值 0-179(OpenCV 8-bit 表示),对光照变化最不敏感;
    \item \textbf{S(Saturation,饱和度)}:颜色的纯度,过曝时趋向 0;
    \item \textbf{V(Value,亮度)}:光照强度。
\end{itemize}

这种分离使得在 H 通道上设置阈值可以较稳定地检测特定颜色,而 S、V 通道用于辅助过滤。

\subsubsection{红色的特殊处理}

在 HSV 空间中,H 通道是\textbf{环形的}:纯红色位于 H=0 附近,同时也延伸到 H=180 附近。因此,简单的 `lower[H] < H < upper[H]` 判断对红色无效。

本项目采用\textbf{wrap 处理}:当检测红色时,分别检测 H $\in$ [0, 10] 和 H $\in$ [170, 180] 两个区间,再取并集。

在工程实现上(\texttt{scripts/utils.py} 的 \texttt{detect\_color\_with\_wrap}),对红色(默认 \texttt{color\_id==0})的环形区间用两种等价方式表达:
\begin{enumerate}
    \item \textbf{“跨零区间”表达}:若配置中 \texttt{H\_lower > H\_upper},则认为区间跨越 0° 边界,自动拆分为 \texttt{[0,H\_upper] $\cup$ [H\_lower,179]} 两段再合并;例如可将红色的 H 写成 \texttt{[173,7]}(\texttt{lower=[173,...]},\texttt{upper=[7,...]});
    \item \textbf{多段并集表达}:在 \texttt{config/colors.json} 中使用 \texttt{hsv\_ranges} 显式写成两段区间(见附录 B)。
\end{enumerate}

注:OpenCV 的 Hue 在 8-bit 表示下通常为 0--179;文中写到 180 主要是便于表述上界,代码中也会把上界按 179/180 处理为同一端点附近的范围。
