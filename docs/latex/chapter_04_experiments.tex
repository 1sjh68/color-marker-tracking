% !TEX root = optics_term_paper.tex

% ==================== 4 实验设计与结果 ====================
\section{实验设计与结果}

\subsection{实验条件}

本节使用最新实验数据 \texttt{output/trajectories/12.16}(三机位 + 标记点姿态重建的后处理结果)进行展示。关键参数与表格来自 \texttt{comparison\_report.txt}:

\begin{table}[htbp]
\centering
\caption{12.16 实验关键参数汇总}
\begin{tabular}{ll}
\toprule
参数 & 值 \\
\midrule
相机数量 & 3(同步拍摄) \\
姿态数据长度 & 485 帧(约 0.87 s) \\
刚体尺寸 & (0.07, 0.19, 0.02) m \\
质量 & 0.1 kg \\
主惯量 ($kg\cdot m^2$) & $I_1=3.04\times10^{-4}, I_2=4.42\times10^{-5}, I_3=3.42\times10^{-4}$ \\
数据表 & \texttt{3d\_trajectories.csv}, \texttt{poses\_opt.csv} \\
\bottomrule
\end{tabular}
\end{table}

下图展示了三视角同步拍摄的实验场景设置,通过 1.00 m 的参考距离标注用于多相机尺度标定:

\begin{figure}[htbp]
\centering
\includegraphics[width=1.0\textwidth]{multi_camera_calibration_scene.png}
\caption{多相机尺度标定场景(三视角下的标记点测量设置与距离参考)}
\end{figure}

为直观展示成像条件,下图给出 12.16 视频的若干原始帧示例,可见光照不均、运动模糊与自遮挡等因素共同构成检测难点:

\begin{figure}[htbp]
\centering
\begin{subfigure}{0.48\textwidth}
    \includegraphics[width=\linewidth]{experiment_frame_000.png}
    \caption{帧 000}
\end{subfigure}
\hfill
\begin{subfigure}{0.48\textwidth}
    \includegraphics[width=\linewidth]{experiment_frame_120.png}
    \caption{帧 120}
\end{subfigure}
\\ \vspace{1em}
\begin{subfigure}{0.48\textwidth}
    \includegraphics[width=\linewidth]{experiment_frame_240.png}
    \caption{帧 240}
\end{subfigure}
\hfill
\begin{subfigure}{0.48\textwidth}
    \includegraphics[width=\linewidth]{experiment_frame_360.png}
    \caption{帧 360}
\end{subfigure}
\caption{12.16 实验原始帧示例(成像条件与遮挡/模糊挑战)}
\end{figure}

此外,为便于论文写作与结果复现,本仓库提供 \texttt{scripts/7\_prepare\_paper\_assets\_12\_16.py}:可将 \texttt{output/trajectories/12.16} 下的关键图表复制到 \texttt{docs/images/},并解析 \texttt{comparison\_report.txt} 与 \texttt{diagnostic\_report.txt} 输出摘要(该脚本不会重新运行三角测量/优化)。

\subsection{重建可视化与 3D 点质量}

姿态重建结果视频为 \texttt{output/trajectories/12.16/reconstruction\_dynamic.mp4},可用于直观观察抛掷过程中的旋转与标记点可见性变化。

为评估三角测量得到的 3D 点质量,\texttt{diagnostic\_report.txt} 给出“每帧完整 3D 点数”的统计(完整指 x/y/z 均非空):

\begin{table}[htbp]
\centering
\caption{每帧有效 3D 标记点数统计(完整帧共 485 帧)}
\begin{tabular}{lll}
\toprule
每帧完整 3D 点数 & 帧数 & 占比 \\
\midrule
0 & 277 & 57.1\% \\
1 & 51 & 10.5\% \\
2 & 157 & 32.4\% \\
3 & 0 & 0.0\% \\
\bottomrule
\end{tabular}
\end{table}

该结果说明:由于遮挡、视角与检测稳定性限制,原始三角测量点云在多数帧是不完整的,因此需要利用刚体约束与时间窗统计/优化进行姿态序列重建。

\begin{figure}[htbp]
\centering
\includegraphics[width=0.9\textwidth]{calibration_candidates.png}
\caption{标定/重建候选帧示意图(用于挑选稳定可见的标记点帧段)}
\end{figure}

\subsection{刚体轨迹与 3D 重建结果}

以 Lab 系中的位移 $(t_x,t_y,t_z)$ 与标记点 3D 点云为基础,可得到刚体在空间中的运动轨迹与姿态可视化结果:

\begin{figure}[htbp]
\centering
\includegraphics[width=1.0\textwidth]{trajectory_3d_reconstruction.png}
\caption{刚体运动轨迹重建(数据驱动 + 物理约束示例结果)}
\end{figure}

\begin{figure}[htbp]
\centering
\includegraphics[width=1.0\textwidth]{reconstructed_3d_plot.png}
\caption{刚体运动轨迹重建 3D 图(用于观察整体平移轨迹与姿态变化)}
\end{figure}

\subsection{姿态与角速度序列}

根据 3.8 的姿态重建流程,可以得到 $\mathcal{R}(t)$(或等价的四元数)与角速度 $\vec\omega_{body}(t)$。下图展示四元数姿态演化:

\begin{figure}[htbp]
\centering
\includegraphics[width=1.0\textwidth]{quaternion_evolution.png}
\caption{姿态四元数分量随时间变化(用于展示姿态连续性与翻转过程)}
\end{figure}

角速度与转动动能的时间序列(由 \texttt{poses\_opt.csv} 计算)如下图所示:

\begin{figure}[htbp]
\centering
\includegraphics[width=1.0\textwidth]{angular_velocity.png}
\caption{体坐标系角速度三分量与转动动能(重建结果示例)}
\end{figure}

\subsection{物理一致性检查:能量、角动量与重力}

姿态/角速度重建完成后,可对守恒量与外场效应进行一致性检查:

\begin{figure}[htbp]
\centering
\includegraphics[width=1.0\textwidth]{angular_momentum_conservation.png}
\caption{角动量守恒性检查(用于检验姿态与角速度的一致性)}
\end{figure}

\begin{figure}[htbp]
\centering
\includegraphics[width=1.0\textwidth]{gravity_analysis.png}
\caption{重力方向上的位移拟合示例(用于检查时间标定与尺度的一致性)}
\end{figure}

\subsection{理论对比:相空间与翻转特征}

基于 \texttt{comparison\_report.txt} 的拟合参数(初始角速度与阻尼系数),可将实验 $\vec\omega(t)$ 与欧拉方程数值积分结果进行对比。对比指标示例:

\begin{itemize}
    \item 角速度平均绝对误差(MAE):$2.2802$ rad/s
    \item 拟合初始角速度:$[0.288, 17.587, 0.159]$ rad/s
    \item 拟合阻尼系数:$[3.91\times10^{-5}, 1.31\times10^{-8}, 1.31\times10^{-5}]$
\end{itemize}



\begin{figure}[htbp]
\centering
\includegraphics[width=1.0\textwidth]{phase_space_comparison.png}
\caption{相空间轨迹对比(基准模拟 vs 实验)}
\end{figure}
