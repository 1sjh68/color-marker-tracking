% !TEX root = optics_term_paper.tex

% ==================== 5 讨论:光学因素与改进建议 ====================
\section{讨论:光学因素与改进建议}

\subsection{主要失败模式归因}

\begin{table}[htbp]
\centering
\caption{视觉测量系统常见失效模式分析}
\begin{tabular}{lll}
\toprule
失败模式 & 光学原因 & 解决方案 \\
\midrule
过曝导致检测失败 & S 饱和度降低,阈值失效 & 多段 HSV + 移除 S 下限硬约束 \\
运动模糊导致拟合失败 & 轮廓拉长变形 & 放宽长宽比阈值至 15.0 \\
背景误检 & 颜色相似区域 & 排除模式 + ROI 框选 \\
噪点被误判为椭圆 & 碎片形状 & 圆度 + 面积比双约束 \\
\bottomrule
\end{tabular}
\end{table}

\subsection{光学端改进建议}

\begin{enumerate}
    \item \textbf{补光}:使用漫射光源(如 LED 柔光灯),减少高光与阴影;
    \item \textbf{偏振}:光源前 + 镜头前加交叉偏振片,抑制镜面反射;
    \item \textbf{材料}:改用哑光贴纸或喷砂膜,减少高光;
    \item \textbf{相机设置}:锁定曝光与白平衡,缩短快门时间以减少模糊。
\end{enumerate}

\subsection{算法端改进建议}

\begin{enumerate}
    \item \textbf{自适应阈值}:根据图像整体亮度动态调整 S/V 范围;
    \item \textbf{运动预测}:引入卡尔曼滤波,利用历史轨迹预测当前位置,减少漏检;
    \item \textbf{相机标定}:进行内参标定与去畸变,提升边缘区域定位精度。
\end{enumerate}

\subsection{系统误差分析}

视觉测量系统的误差可分为以下几类:

\subsubsection{随机误差}

\begin{table}[htbp]
\centering
\caption{系统随机误差来源分析}
\begin{tabular}{lll}
\toprule
误差来源 & 典型量级 & 影响 \\
\midrule
椭圆拟合中心定位 & 0.3–1.0 像素 & 直接影响 2D 坐标精度 \\
形态学处理导致轮廓偏移 & 1–2 像素 & 影响椭圆拟合质量 \\
图像量化噪声 & 0.5 像素 & 在差分运算中被放大 \\
帧间检测跳点 & 偶发 & 通过时空连续性过滤抑制 \\
\bottomrule
\end{tabular}
\end{table}

\subsubsection{系统误差}

\begin{enumerate}
    \item \textbf{相机标定误差}
    \begin{itemize}
        \item 未进行内参标定时,镜头径向畸变会导致图像边缘区域的标记定位存在 \textbf{1–3 像素}的系统偏移;
        \item 外参标定不精确会导致多视图三角测量的重投影误差增大。
    \end{itemize}
    \item \textbf{透视投影误差}
    \begin{itemize}
        \item 标记点直径在图像中的投影尺寸随距离变化:$d_{px} = fD/Z$;
        \item 当物体距相机距离变化较大时,需考虑尺度修正。
    \end{itemize}
    \item \textbf{时间同步误差}
    \begin{itemize}
        \item 多相机采集若无硬件同步,帧间时间偏差可达 \textbf{1–2 帧};
        \item 对于 60 fps 的视频,这相当于 \textbf{16–33 ms} 的时间偏移,在高速旋转场景中会导致显著的姿态误差。
    \end{itemize}
    \item \textbf{物体形状偏离假设}
    \begin{itemize}
        \item 若刚体质量分布不均匀,实际主惯量轴与几何主轴存在偏差;
        \item 这种偏差在动力学验证中会表现为理论与实验的系统性差异。
    \end{itemize}
\end{enumerate}

\subsubsection{误差传递与累积}

在从像素坐标到物理量(如角速度)的计算链路中,误差会逐级传递并可能放大:
\begin{equation}
\delta\omega \sim \frac{\delta\theta}{\Delta t} \sim \frac{\delta p / L}{\Delta t}
\end{equation}
其中 $\delta p$ 为像素定位误差,$L$ 为特征臂长(像素),$\Delta t$ 为帧间隔。对于典型参数($\delta p = 1$ px,$L = 100$ px,$\Delta t = 1/60$ s),角速度误差约为 $\delta\omega \sim 0.6$ rad/s,这与实验中观测到的角速度 MAE(约 2.28 rad/s)同阶。
