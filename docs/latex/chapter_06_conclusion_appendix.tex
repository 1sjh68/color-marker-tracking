% !TEX root = optics_term_paper.tex

% ==================== 6 结论与展望 ====================
\section{结论与展望}

\subsection{结论}

本文设计并实现了一套基于彩色标记点的高速视觉测量系统。通过将问题分解为光学成像链路与算法处理链路,系统性地分析了影响检测精度的关键因素,并在算法端实现了多段 HSV 并集、排除区域、增强型椭圆拟合与时空连续性过滤等机制。以 12.16 三机位数据为例,本文展示了从 2D 检测到 3D 点质量诊断、连续姿态/角速度序列重建的结果,并通过角动量、重力拟合等指标进行一致性检查;在拟合带阻尼欧拉方程后,理论与实验角速度的 MAE 约为 2.28 rad/s,能量耗散趋势与预期一致。

\subsection{展望}

未来工作方向包括:
\begin{enumerate}
    \item \textbf{统一可复现实验流水线}:将 12.16 的三角测量、窗口优化姿态求解与理论对比脚本整合到本仓库的一条命令行流程中(从视频/CSV 到 \texttt{poses\_opt.csv} 与全部图表一键生成)。
    \item \textbf{时间标定与尺度一致性}:从视频元数据/慢放倍率自动推断等效 FPS,并用重力拟合等外部参照交叉校准;同时减少对经验性 \texttt{Z\_SCALE} 修正的依赖。
    \item \textbf{几何标定与联合优化}:完善多相机内外参标定与 bundle adjustment,将“刚体约束/重投影误差/时间连续性”纳入同一优化框架,提高 3D 点完整性与姿态精度。
    \item \textbf{检测鲁棒性与实时性}:引入运动预测(卡尔曼/粒子滤波)、更稳健的特征(颜色+几何/Tag 混合),并通过并行化/硬件加速向实时重建推进。
\end{enumerate}

% ==================== 参考文献 ====================
\begin{thebibliography}{99}
\bibitem{opencv} OpenCV Documentation. \textit{cv::fitEllipse}, \textit{cv::findContours}, \textit{cv::cvtColor}. https://docs.opencv.org/
\bibitem{hartley} Hartley R, Zisserman A. \textit{Multiple View Geometry in Computer Vision}. Cambridge University Press, 2003.
\bibitem{gonzalez} Gonzalez R C, Woods R E. \textit{Digital Image Processing}. Pearson, 2018.
\bibitem{born} Born M, Wolf E. \textit{Principles of Optics}. Cambridge University Press, 1999.
\bibitem{szeliski} Szeliski R. \textit{Computer Vision: Algorithms and Applications}. Springer, 2010.
\end{thebibliography}

\appendix

% ==================== 附录 A ====================
\section{算法参数表}

下表中未标注的参数对应本仓库的 2D 检测/跟踪与像素域后处理;带 “(12.16)” 的参数来自配套的多相机姿态重建流程(用于生成 \texttt{output/trajectories/12.16/poses\_opt.csv})。

\begin{longtable}{llll}
\caption{视觉测量系统算法关键参数表} \\
\toprule
参数类别 & 参数名 & 值 & 说明 \\
\midrule
\endfirsthead
\toprule
参数类别 & 参数名 & 值 & 说明 \\
\midrule
\endhead
\bottomrule
\endfoot
\textbf{形态学} & Close Kernel & (9, 9) & 闭运算核,填充空洞 \\
 & Open Kernel & (5, 5) & 开运算核,去除噪点 \\
 & Close Iterations & 2 & 闭运算迭代次数 \\
 & Open Iterations & 1 & 开运算迭代次数 \\
\midrule
\textbf{椭圆筛选} & MIN\_AREA & 100 & 最小轮廓面积(像素) \\
 & MAX\_ASPECT\_RATIO & 15.0 & 最大长宽比 \\
 & MIN\_CIRCULARITY & 0.3 & 最小圆度 \\
 & MIN\_AREA\_RATIO & 0.5 & 最小面积匹配度 \\
\midrule
\textbf{后处理} & NMS\_DISTANCE & 30 & NMS 重复检测抑制距离(像素) \\
\midrule
\textbf{轨迹过滤} & max\_frame\_gap & 3 & 允许的最大帧间断(遮挡/漏检容忍) \\
 & max\_spatial\_jump & 50 px & 相邻帧最大空间跳变 \\
 & min\_segment\_length & 5 & 最短连续段长度 \\
\midrule
\textbf{运动筛选} & motion\_threshold & 5.0 px & 位置标准差阈值(抑制静态干扰) \\
 & min\_points & 50 & 轨迹最少点数(连续性要求) \\
\midrule
\textbf{像素域拟合} & smooth\_window & 11 & 中心轨迹滑动平均窗口(默认) \\
\midrule
\textbf{姿态优化(12.16)} & WINDOW\_SIZE & 20 & 时间窗半宽 ($\pm$20 帧) \\
 & loss & soft\_l1 & 鲁棒损失 (\texttt{least\_squares}) \\
 & f\_scale & 0.05 & 软阈值尺度(影响鲁棒性) \\
 & max\_nfev & 50 & 单帧最大迭代次数 \\
\midrule
\textbf{异常剔除(12.16)} & max\_angle\_jump & 0.5 rad & 相邻帧旋转跳变阈值 \\
 & max\_pos\_jump & 0.1 m & 相邻帧平移跳变阈值 \\
\midrule
\textbf{角速度(12.16)} & MAX\_OMEGA & 10 rad/s & 角速度模长上限(抑制异常) \\
 & median\_window & 5 & 中值滤波窗口(去尖峰) \\
 & gaussian\_sigma & 2.0 & 高斯平滑参数(抑制高频) \\
\end{longtable}

% ==================== 附录 B ====================
\section{HSV 范围配置示例}

\texttt{config/colors.json} 以 JSON 形式保存颜色与阈值范围。除 \texttt{hsv\_lower/hsv\_upper} 外,还支持:
\begin{itemize}
    \item \texttt{hsv\_ranges}:同一颜色的多段范围(并集)
    \item \texttt{hsv\_excludes}:排除范围(从掩码中减去)
    \item \texttt{roi}:可选 ROI($[x_1, y_1, x_2, y_2]$)
\end{itemize}

\subsection{HSV 通道的光学意义}

\begin{table}[htbp]
\centering
\caption{HSV 通道物理意义与标定策略}
\begin{tabular}{p{0.2\textwidth}p{0.25\textwidth}p{0.1\textwidth}p{0.15\textwidth}p{0.2\textwidth}}
\toprule
通道 & 物理意义 & 范围 & 光照敏感性 & 标定策略 \\
\midrule
\textbf{H (Hue)} & 光谱主波长,对应色调 & 0--179 & 低 & 目标色为中心 $\pm$10--15 \\
\textbf{S (Saturation)} & 颜色纯度 & 0--255 & 高 & 设下限除灰白,过曝可降至50 \\
\textbf{V (Value)} & 亮度 & 0--255 & 高 & 随光照动态调整,阴影区V低 \\
\bottomrule
\end{tabular}
\end{table}

\subsection{示例标定参数}

以下为本仓库当前配置文件中的一组 HSV 阈值示例:

\begin{table}[htbp]
\centering
\caption{实测各标记点颜色阈值区间 (12.16 实验)}
\begin{tabular}{lllll}
\toprule
颜色 & H 范围 & S 范围 & V 范围 & 备注 \\
\midrule
\textbf{红色 (ID=0)} & [0, 7] $\cup$ [173, 179] & [132, 228] & [51, 171] & 环形区间 \\
\textbf{绿色 (ID=1)} & [57, 57] & [135, 170] & [80, 227] & 含并在范围 \\
\textbf{蓝色 (ID=2)} & [103, 127] & [91, 191] & [25, 145] & 随光照调整 \\
\textbf{黄色 (ID=3)} & [15, 39] & [187, 255] & [79, 199] & 高饱和要求 \\
\bottomrule
\end{tabular}
\end{table}

\subsection{配置文件示例}

\begin{lstlisting}[caption={config/colors.json 配置片段}]
{
  "colors": [
    {
      "id": 0,
      "name": "Red",
      "bgr": [0, 0, 255],
      "hsv_lower": [0, 132, 51],
      "hsv_upper": [7, 228, 171]
    },
    {
      "id": 1,
      "name": "Green",
      "bgr": [0, 255, 0],
      "hsv_lower": [57, 135, 80],
      "hsv_upper": [57, 170, 227],
      "hsv_ranges": [
        {"lower": [57, 135, 80], "upper": [57, 170, 227]},
        {"lower": [27, 71, 0], "upper": [51, 191, 152]}
      ]
    }
  ]
}
\end{lstlisting}

\subsection{多段 HSV 范围的必要性}

在实际拍摄中,同一颜色标记点在不同光照条件下的 HSV 值可能差异显著:
\begin{itemize}
    \item \textbf{迎光面}:V 值较高(可达 200+),S 值因高光可能下降;
    \item \textbf{背光面}:V 值较低(低至 50–80),但 S 值相对稳定;
    \item \textbf{运动模糊}:模糊拖影区域的颜色饱和度降低,H 值向灰色方向偏移。
\end{itemize}
因此,单一的 \texttt{[hsv\_lower, hsv\_upper]} 范围往往无法覆盖所有情况,本系统采用的\textbf{多段范围并集策略}能够有效应对这些光照变化。

% ==================== 附录 C ====================
\section{单目相机尺度标定原理}

在单目视觉系统中,若需要从 2D 像素坐标恢复物理尺度,必须标定相机的焦距与尺度关系。本系统采用一种基于\textbf{已知物理尺寸}的交互式标定方法。

\subsection{标定原理}

根据针孔相机模型,物体上两点在图像中的像素距离 $d_{px}$ 与其物理距离 $L$ 及深度 $Z$ 的关系为:
\begin{equation}
d_{px} = \frac{f \cdot L}{Z - Z_0}
\end{equation}
其中 $Z_0$ 为相对于参考系原点的光心平移。将其改写为关于 $1/d_{px}$ 的线性方程:
\begin{equation}
\frac{1}{d_{px}} = \left( \frac{1}{f \cdot L} \right) Z - \frac{Z_0}{f \cdot L}
\end{equation}
令 $Y = 1/d_{px}$,$X = Z$,则有 $Y = m X + c$。

\subsection{标定流程}

\begin{enumerate}
    \item \textbf{选择标定帧}:在视频中选择多个不同深度的帧(物体距相机距离变化)
    \item \textbf{测量像素距离}:在每帧中点击标记点对应的两个端点,计算像素距离 $d_{px}$
    \item \textbf{获取深度值}:从 3D 重建结果中获取对应帧的 Z 坐标
    \item \textbf{线性回归}:对 $(Z, 1/d_{px})$ 数据点进行最小二乘拟合
    \item \textbf{提取参数}:
    \begin{itemize}
        \item $f \cdot L = 1/m$
        \item $Z_0 = c/m$
    \end{itemize}
\end{enumerate}

\subsection{实验标定结果}

以 12.16 数据集为例,使用刚体对角线长度 $L = \sqrt{0.19^2 + 0.07^2} \approx 0.2025$ m 作为标定基准:
\begin{itemize}
    \item 拟合斜率 $m \approx 0.002$ ($1/(f \cdot L)$)
    \item 拟合截距 $c \approx 0.001$ ($-Z_0/(f \cdot L)$)
    \item 判定系数 $R^2 > 0.95$
\end{itemize}
据此可推算出 $f \cdot L \approx 500$ 像素$\cdot$米,$Z_0 \approx 0.5$ 米。这为后续的 3D 深度校正提供了物理依据。

\subsection{标定质量评估}

标定结果的质量可通过以下指标评估:
\begin{enumerate}
    \item \textbf{拟合残差}:线性回归的 R² 值应 > 0.95
    \item \textbf{重投影误差}:使用标定参数预测的像素距离与实测的误差应 < 5 像素
    \item \textbf{物理合理性}:$f \cdot L$ 的值应与相机规格和物体尺寸相符
\end{enumerate}

% ==================== 附录 D ====================
\section{系统性能指标}

基于实际测试,本视觉测量系统的典型性能指标如下:

\begin{table}[htbp]
\centering
\caption{系统运行效能统计}
\begin{tabular}{lll}
\toprule
指标 & 典型值 & 测试条件 \\
\midrule
\textbf{检测率} & 95\%+ & 良好光照,标记点尺寸 $> 20$ 像素 \\
\textbf{降噪率} & 90\%+ & 时空连续性过滤 + 运动幅度筛选 \\
\textbf{定位精度} & 亚像素级 (0.3--1.0 px) & 椭圆拟合中心定位 \\
\textbf{处理速度} & $\sim$200 fps & CPU 单线程,1080p 视频 \\
\textbf{角速度精度} & $\pm$0.6 rad/s & 60 fps,100 px 特征臂长 \\
\bottomrule
\end{tabular}
\end{table}

% ==================== 附录 E ====================
\section{项目代码仓库}

本文所述视觉测量系统的完整源代码已开源至 GitHub:

\begin{center}
\url{https://github.com/1sjh68/color-marker-tracking}
\end{center}
。
